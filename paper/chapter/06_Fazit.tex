% !TEX root = ../Thesis.tex
%%
%%  Hochschule für Technik und Wirtschaft Berlin --  Abschlussarbeit
%%
%% Kapitel 6
%%
%%

\chapter{Fazit} \label{Fazit}

In diesem Kapitel wird die erstellte Lösung begutachtet. Es werden Anwendungsmöglichkeiten, allgemein von Smart Objects und speziell für die Implementierung, behandelt. Mögliche Erweiterungen werden vorgeschlagen und ein Fazit der Arbeit wird gezogen.

\section{Analyse der Implementierung}

Die erstellte Lösung wird kritisch begutachtet. Dafür werden zuerst Kostenbetrachtungen bezogen auf die Hardware und auf die Software durchgeführt. Die Ergebnisse fließen in die abschließende Bewertung mit ein. 

\subsection{Kostenbetrachtung}

Um in der Bewertung der Implementierung die Kosten einschätzen zu können, sollen hier zum einem die Hardwarekosten vorgestellt werden. Dazu sollen die einzelnen Materialkosten der verwendeten Bauteile zusammengefasst werden. Um die gesamten Hardwarekosten ermitteln zu können, müssen auch Entwicklungskosten und Fertigungskosten berücksichtigt werden. Fertigungskosten beinhalten dabei die eigentlichen Arbeitsstunden, die für die Fertigung aufgebracht werden müssen, als auch die Investitionskosten für den Fertigungsarbeitsplatz. Weil das sehr abhängig von der Art der Fertigung ist, sei es eine Einzelstückfertigung oder eine Serienfertigung, werden diese Kosten hier nicht näher betrachtet. Die Hardwarekosten werden rein durch die Materialkosten repräsentiert.

Zum anderen sollen die Softwarekosten abgeschätzt werden. Dies wird durch die Ermittlung der Entwicklungskosten für die Contiki-Applikation iec104 erreicht. Anhand eines angenommenen Stundensatzes für einen Softwareentwickler und den aufgewendeten Stunden werden die Kosten für die Entwicklung berechnet.

\subsubsection{Materialkosten}

In Tabelle \ref{tab:hardwarekosten} sind die Bauteile aufgelistet, die zum Erstellen der Steckdose verwendet wurden. Sie benennt das Bauteil und zeigt die Anzahl an, die verbaut wurden. Die Bauteile wurden -- bis auf die Tchibo Steckdose -- vom Handelsunternehmen Mouser Electronics bezogen. In der dritten Spalte ist die entsprechende Artikelnummer aufgelistet. Die vierte Spalte zeigt den Preis bei dem Bezug von nur einem Bauteil. Bei einer Massenfertigung der Steckdose werden andere Preise angeboten. Der Stückpreis bei einer Bestellung von 1000 Fertigungssätzen ist in der letzten Spalte aufgelistet. Die Preise wurden am 06.05.2013 abgefragt und sind inklusive Mehrwertsteuer angegeben.

\begin{table}[btp]
\small
\caption{Materialkosten Steckdose}
\begin{tabular}{|r|l|l|p{2cm}|p{2cm}|}\hline
  \textbf{\#} & \textbf{Bauteil} & \textbf{Artikelnummer} & \textbf{Einzel-preis in~€} & \textbf{Massen-preis in~€} \\ \hline
  \multicolumn{5}{|l|}{Zigbit-Platine} \\ \hline
  1 & ATZB-24-A2 & 556-ATZB-24-A2 & 25,43 & 14,42 \\ \hline
  1 & Widerstand 100kOhm & 71-CMF60100K00FKEB & 0,206 & 0,099 \\ \hline
  1 & Widerstand 1kOhm & 71-CMF551K0000FHEK  & 0,116 & 0,005 \\ \hline
  1 & Kondensator 3,3uF & 667-EEU-HD1H3R3 & 0,165 & 0,104 \\ \hline
  1 & Spannungsregler LP2950CZ-3.0 & 926-2950CZ-3.0/NOPB & 0,676 & 0,27 \\ \hline
  1 & Steckverbinder FFC 6 Pin & 538-52271-0679 & 1,30 & 0,583 \\ \hline
  2 & Steckverbinder FFC 18 Pin & 538-52271-1879 & 1,71 & 0,94 \\ \hline
  \multicolumn{5}{|l|}{Steckdosen-Platine} \\ \hline
  1 & Kondensator X2 275V & 80-R46KN368050M2M & 0,66 & 0,263 \\ \hline
  1 & Widerstand 560Ohm, 5Watt & 594-AC05W560R0J & 0,38 & 0,198 \\ \hline
  2 & Widerstand 1MOhm & 594-MRS251M1\%TR & 0,074 & 0,038 \\ \hline
  2 & Gleichrichterdiode & 512-1N4004 & 0,076 & 0,025 \\ \hline
  2 & Zener-Diode 24V & 512-1N4749ATR & 0,203 & 0,036 \\ \hline
  1 & Kondensator 470uF & 667-EEU-FR1E471YB & 0,248 & 0,152 \\ \hline
  1 & NPN-Transistor BC547B & 512-BC547B & 0,186 & 0,05 \\ \hline
  1 & Widerstand 220kOhm & 271-220K-RC & 0,125 & 0,012 \\ \hline
  \multicolumn{5}{|l|}{Steckdosen-Gehäuse} \\ \hline
  1 & Tchibo Digitale Zeitschaltuhr & 4 043002 669758 & 5,99 & 5,99 \\ \hline
  \multicolumn{5}{|l|}{} \\ \hline
  \multicolumn{3}{|l|}{\textbf{Gesamtpreis}} & \textbf{38,61} & \textbf{24,27} \\ \hline
\end{tabular}
\label{tab:hardwarekosten}
\end{table}

Die Gesamtmaterialkosten der verwendeten Bauteile betragen 38,61€. Nicht berücksichtigt sind Verbrauchsmaterialien. Dazu zählen die Leiterplatte für die Zigbit-Platine, die aus einer vorhandenen Lochrasterplatine ausgesägt wurde, Leitungen für die Verkabelung und Lötzinn. Diese Kosten sollen hier vernachlässigt werden. Bei einer Kalkulation für eine Fertigung in einem Unternehmen müssen diese als Materialgemeinkostenzuschlag zu den Materialkosten hinzugefügt werden. Zusammen mit den Fertigungskosten, also die Arbeitskosten, die zum Fertigen der Steckdose notwendig sind, und den Verwaltungs- und Vertriebsgemeinkostenzuschlag ergeben sich die Selbstkosten je Stück.



\subsubsection{Entwicklungskosten iec104}

Um den Aufwand von einer Contiki-Applikation abzuschätzen, sollen beispielhaft die Entwicklungskosten der implementierten Applikation iec104 ermittelt werden. Zum Ermitteln der Kosten, muss der Stundensatz des Entwicklers mit der Zeit in Stunden multipliziert werden, die dafür notwendig ist, eine solche Applikation zu programmieren. Dazu gehört eine Designphase, die eigentliche Programmierung und eventuelle Fehlerkorrekturen. Weil die Kosten proportional zur Zeit sind, soll nur diese betrachtet werden. Es wird geschätzt, dass für die Entwicklung der Applikation iec104 etwa drei bis vier Mannwochen benötigt wurden. Die Schätzung wird dadurch erschwert, dass nicht kontinuierlich an der Entwicklung gearbeitet wurde. Die Arbeit daran wurde öfter durch andere notwendige Arbeiten unterbrochen.

Deswegen soll noch eine quantitative Bewertung anhand der Anzahl geschriebener Zeilen Quellcode LOC (Lines Of Code) stattfinden. In der Praxis ist solch eine Bewertung, also der Rückschluss von den LOC auf den Aufwand, allerdings problematisch. Die Zeit, die für die Designphase verwendet wird, wird nicht berücksichtigt. Eine überlegte und optimierte Programmierung benötigt tendenziell weniger LOC. Die Qualität der Programmierung wird auch nicht bewertet. Verschiedene Programmiersprachen und individuelle Programmierstile wirken sich ebenso auf die LOC aus. 

Trotzdem gibt es Faustformeln, die einen Zusammenhang zwischen der Anzahl geschriebener Zeilen Quellcode LOC und der eingesetzten Entwicklungszeit sehen. Nach \cite[Seite 305]{Ludewig:SoftwareEngineering} wird in einem Softwareprojekt nach n Stunden Aufwand -- wobei hier der Gesamtaufwand gemeint ist, nicht die reine Programmierungszeit -- ein System mit zwei mal n LOC erzeugt. Ein studentisches Projekt, das nicht kommerziell vertrieben werden soll, kann in der gleichen Zeit in etwa die dreifache Anzahl an LOC liefern. Dort können demnach mit jeder Stunde Aufwand in etwa sechs LOC entwickelt werden.

Um nach dieser Faustformel herauszufinden, wie viele Stunden in die Entwicklung der Contiki-Applikiation iec104 eingeflossen sind, werden die Anzahl an Zeilen Quellcode ermittelt. Die Contiki-Applikation iec104 besteht aus 14 Dateien:
\begin{itemize}
	\itemsep 0pt
	\item 1 Makefile
	\item 7 Header-Dateien
	\item 6 C-Dateien
\end{itemize}

Insgesamt enthalten diese Dateien 1011 Zeilen. Wenn die 211 Leerzeilen und 112 Kommentarzeilen abgezogen werden, bleiben 688 Zeilen Quellcode. Nach der Faustformel oben ergibt sich eine Entwicklungszeit von etwa 115 Stunden. Bei der Annahme von einem Stundensatz von 100€ ergeben sich Entwicklungskosten von 11.500€. Bei einer Wochenarbeitszeit von 40 Stunden, ergeben sich ungefähr 3 Mannwochen, die notwendig sind, um die Contiki-Applikation iec104 zu entwickeln. Das deckt sich in etwa mit den Erfahrungen aus dieser Arbeit.



\section{Anwendungsmöglichkeiten}

Dieses Kapitel stellt zuerst allgemein verschiedene Anwendungsbereiche von Smart Objects vor. Darauffolgend wird erörtert, in welchen Anwendungsbereichen die erstellte Lösung eingesetzt werden könnte. Anhand der Erfahrung, die bei der Implementierung gemacht wurden, werden mögliche Erweiterungen behandelt.

\subsection{Anwendungsbereiche von Smart Objects}

Smart Objects können in einer Vielzahl von Anwendungsbereichen eingesetzt werden, da sie sehr flexibel in ihrer Beschaffenheit sind. Die Programmierung kann individuell angepasst werden. Durch die Verwendung der richtigen Sensoren und Aktoren sind sie für verschiedenste Einsatzbereiche verwendbar. Nachfolgend sollen folgende fünf mögliche Anwendungsbereiche kurz beschrieben werden:

\begin{itemize}
	\itemsep 0pt
	\item eHome-Bereich
	\item Gebäudeautomation
	\item Industrieautomatisierung
	\item Logistik
	\item Smart Grid
\end{itemize}

Der eHome-Bereich ist ein Bereich in dem sich mehrere proprietäre Lösungen verbreitet haben, die oft nicht untereinander kompatibel sind. Dies kann auch ein Mitgrund dafür sein, dass sich die Heimautomatisierung bisher nicht so entwickelt hat wie vor etwa 7-9 Jahren prognostiziert \citep[360]{vasseur10interconnecting}. Wichtig ist auch, gerade im eHome-Bereich, eine einfache Installation der Geräte. Nur wenn ein Endnutzer ohne spezielles Expertenwissen Geräte in Betrieb nehmen und verwenden kann, wird sich eine Lösung durchsetzen können. Mögliche Einsatzgebiete von Smart Objects im eHome-Bereich sind zum Beispiel Steuerungen von Licht, der Heizung, von Fenster, von Rollläden und von Türschlössern.

Der eHome-Bereich wird für private Wohnhäuser eingesetzt. Im Gegensatz dazu zielt die Gebäudeautomation eher auf den professionellen Einsatz in großen Gebäuden meist im Zusammenspiel mit einem visualisierten Gebäudemanagementsystem. \textcite[Seite 304ff]{Vermesan:TheInternetOfThings} beschreiben unter dem Titel "`Smart IPv6 Building"' verschiedene Forschungsprojekte, die die Verwendung von IPv6 in der Gebäudeautomation untersuchen. Unter anderem wird auch das Hobnet-Projekt erwähnt, das spezielle Anwendungsfälle von Smart Objects in der Gebäudeautomation untersucht. Hier kommt auch 6LoWPAN zum Einsatz. Allgemein sind die Ziele einer Gebäudeautomation Energieeinsparungen, Sicherheit und Komfortgewinn.


Der Bereich Smart Grid wird einer der größten Anwendungsbereiche für Smart Objects werden \citep[Seite 271ff]{Hersent:TheInternetOfThings}. Seit mehreren Jahren geht der Trend in der Stromerzeugung immer mehr in Richtung erneuerbaren Energiequellen wie Windkraftanlagen oder Photovoltaikanlagen. Diese sind aber im Gegensatz zu klassischen fossilen Kraftwerken dezentral aufgestellt. Das führt zu einem erheblichen Umbruch in der Stromnetzführung. Die Energie wird nicht allein an wenigen zentralen Örtlichkeiten durch große Kraftwerke, sondern auch dezentral durch kleine teils privat teils kommerziell geführten Energiequellen erzeugt. Das erschwert die Kraftwerksregelung und die Leitung des Lastflusses. Um trotzdem eine gute Netzstabilität zu gewährleisten, muss das Stromnetz intelligenter und vielfältiger überwacht und gesteuert werden.

Eine Maßnahme dabei ist das Smart Metering. Stromverbrauchszähler liefern über eine Kommunikationsschnittstelle den aktuellen Energieverbrauch an das Energieversorgungsunternehmen. Solche Art intelligente Zähler existieren schon länger für große Energieverbraucher wie Produktionsbetriebe, seit einigen Jahren werden Smart Meter auch für Privathaushalte angeboten.


\subsection{Anwendungsmöglichkeiten für die Implementierung}


Das Protokoll IEC104 ist ein Fernwirkprotokoll aus dem Bereich der Energieautomatisierung. Der Bereich Smart Grid ist also ein klassischer Anwendungsfall für dieses Protokoll. An ein zentrales Leitsystem werden Informationen übertragen und Steuerbefehle entgegen genommen. Allerdings wird der Zugriff auf eine Steuerung von einem elektrischen Verbraucher im Privathaushalt für ein Energieversorgungsunternehmen nicht von Bedeutung sein. Ein möglicher Anwendungsfall liegt eher in einem Smart Meter. Hier könnten mittels IEC104 Verbrauchsstände übertragen werden. Zusätzlich wäre in Absprache mit dem Privathaushalt eine Steuerung möglich. Genaue Anwendungsfälle müssen noch erprobt werden. Denkbar wäre eine Notabschaltung einer lokalen Photovoltaikanlage, wenn mehr Energie ins Stromnetz eingespeist wird als dort verbraucht wird bzw. in andere Stromnetzregionen abgeführt werden kann. Ebenfalls denkbar ist eine gezielte Steuerung von größeren Energieverbrauchern, bei denen der Einsatz zeitlich flexibel ist (Demand Side Management). Beispiele dafür sind die Waschmaschine oder die Batterie eines Elektrofahrzeugs. Bei einer großer Netzauslastung können diese steuerbaren Verbraucher zurückgefahren werden. Das Energieversorgungsunternehmen muss dann weniger Reserven für Spitzenlast vorhalten. So hilft eine intelligente Netzführung dabei, Kosten in der Energieerzeugung zu reduzieren.

\subsection{Erweiterungsvorschläge für die Implementierung}

Die erstellte Lösung ist ein Prototyp, der beispielhaft implementiert worden ist. Während der Arbeit sind mehrere Ideen entstanden, wie die Implementierung verbessert oder erweitert werden kann. Zuerst werden Vorschläge für die IEC104-Slave-Applikation aufgelistet.

\begin{itemize}
	\itemsep 0pt
	\item Informationsmeldungen mit Zeitstempel
	\\Die Statusänderung der Steckdose erfolgt aktuell über den Datentyp M\_SP\_NA\_1 (Type Ident 1): Einzelbitmeldung ohne Zeitstempel. Eine Verbesserung wäre die Verwendung des Datentyps M\_SP\_TB\_1 (Type Ident 30): Einzelbitmeldung mit dem Zeitstempel CP56Time2a. Hier wird zusätzlich ein Zeitstempel mit Jahr, Monat, Tag, Stunde, Minute, Sekunde und Millisekunde übertragen. So ist der exakte Zeitpunkt der Statusänderung bekannt. Voraussetzung dafür ist aber, dass die Steuerung mit der aktuellen Zeit synchronisiert ist. Über IEC104 existiert dafür eine Möglichkeit mit dem Datentyp C\_CS\_NA\_1 (Type Ident 103): Zeitsynchronisationsbefehl. Ein Zeitstempel wird vom IEC104-Master zum IEC104-Slave gesendet und dort übernommen. Weil die Übertragungszeit dabei aber nicht berücksichtigt wird, ist die Verwendung vom IEC-Standard nur bedingt empfohlen. In der Praxis wird oft das Protokoll NTP (Network Time Protocol) verwendet. Eine Implementierung von NTP im Betriebssystem Contiki existiert bereits \cite{Contiki-syslog}.
	\item interne Statusinformationen des Contiki Betriebssystems
	\\Das Modul iec104-para kann mit zusätzlichen Informationsobjekten erweitert werden. Zusätzliche interne Statusinformationen wie die Anzahl der laufenden Prozesse, TCP/IP-Verbindungen oder die Betriebszeit könnten übertragen werden. Für Messwerte gibt es bereits den Datentyp M\_ME\_NA\_1 (Type Ident 9): Messwert, normalisiert und ohne Zeitstempel. Weitere Datentypen können implementiert werden.
\end{itemize}

Eine weitere Verbesserungsmöglichkeit betrifft nicht die Implementierung selbst sondern dem Testaufbau. Als 6LoWPAN-Router wurde ein handelsüblicher PC mit einer Linux-Installation verwendet. Als 6LoWPAN-Schnittstelle wurde ein USB-Stick RZUSBSTICK von Atmel verwendet, der auf dem PC eine Ethernet-Schnittstelle simuliert. Unter anderem hat das den Nachteil, dass auf dem PC der 6LoWPAN-Netzwerkverkehr nicht beobachtet werden kann, weil nur der simulierte Ethernet-Verkehr sichtbar ist. Es wird aber aktuell daran gearbeitet, 6LoWPAN direkt im Linux-Kernel zu implementieren \citep{Ott2012}. Eine eingeschränkte Variante wird schon seit der Kernel Version 3.2.46 unterstützt und laufend erweitert. Als Funkchips werden der MRF24J40 von Microchip und der AT86RF230 von Atmel, der auch im Zigbit-Modul verbaut ist, unterstützt. \citeauthor{Ott2012} stellt eine Hardware vor bestehend aus einem BeagleBone und einem MRF24J40MA Funkchip. Damit kann 6LoWPAN nativ unter Linux verwendet werden ohne zusätzliche Hardware und ohne ein zusätzliches Betriebssystem. Es ist allerdings nicht bekannt, ob es Interoperabilitätsprobleme zwischen der 6LoWPAN-Implementierung im Linux-Kernel und im Contiki-Betriebssystem gibt.

\section{Zusammenfassung}

Es konnte gezeigt werden, dass eine internetfähige Steuerung mithilfe eines 8-bit Mikrocontrollers implementiert werden kann. Das Betriebssystem Contiki stellt dazu den notwendigen TCP/IP-Stack und andere Werkzeuge und Hilfsmittel zur Verfügung. Eine Webserver-Anwendung und eine Vielzahl anderer Anwendungen sind Teil des Betriebssystems. Mit der Contiki-Applikation iec104 wurde gezeigt, dass die Implementierung des Protokolls IEC104 in einer limitierten Umgebung durchaus möglich ist. Auch die Verwendung von IEC104 im Zusammenspiel mit IPv6 hat funktioniert. Bisher war keine Implementierung von IEC104 über IPv6 bekannt. Generell zeigt die Contiki-Applikation iec104, dass die Implementierung neuer internetfähiger Anwendungen möglich und nicht aufwendiger als bei anderen Betriebssystemen ist. Die geringen Ressourcen müssen allerdings bei der Programmierung berücksichtigt werden.

Bei der Implementierung ist der 8KByte große RAM-Speicher die markanteste Ressourcengrenze. Die Auslastung beträgt 88,9\%. Das Hinzufügen von zusätzlichen Contiki"=Applikationen oder das Erweitern der Applikation iec104 ist deswegen nicht ohne Weiteres möglich. Durch den Einsatz von anderer Hardware mit einem größeren RAM-Speicher kann dieses Problem umgangen werden. So steht bei dem AVR Ravenboard mit 16KByte doppelt so viel RAM-Speicher zur Verfügung. Der Hersteller Redwire bietet mit dem Econotag ein fertiges Modul mit USB-Schnittstelle an. Verwendet wird der Freescale MC13224V, der einen 32-bit ARM7 Mikrocontroller mit einer IEEE-802.15.4-Schnittstelle kombiniert. Er verfügt über 128KByte Flash- und 96KByte RAM-Speicher. Das Betriebssystem Contiki unterstützt diese Modul über die Platform redbee-econotag. Beide, das AVR Ravenboard und das Econotag, sind aber für den Einbau in die verwendete Steckdose zu groß.

%\cite{ko11beyond}: In this paper we present two interoperable implementations of the IPv6 protocol stack for LLNs for two leading sensor network operating systems: Contiki and TinyOS. We demonstrate that the two implementations interoperate, but also show that interoperability is not enough. Rather, we expose the fact that subtle differences in different layers of the protocol stack can affect the resulting system performance. 

Als Kommunikationstechnologie wurde 6LoWPAN verwendet. Es scheint ideal für den Einsatz bei ressourcenbeschränkten Geräten zu sein. Anfang dieses Jahres hat die ZigBee Alliance die Spezifikation ZigBee IP veröffentlicht \citep{zigbee-ip}. Damit werden unter der Verwendung von 6LoWPAN vermaschte drahtlose IPv6-Netzwerke unterstützt. Dies ist ein Beispiel dafür, dass die Verwendung von 6LoWPAN in vielen Bereichen voranschreitet. Eine weitere Entwicklung, die die Verbreitung von 6LoWPAN unterstützt, ist die Spezifizierung des Protokolls RPL \citep{RFC6550}. RPL ist ein Routing-Protokoll, das für verlustbehaftete Sensornetze entwickelt worden ist. Die speziellen Anforderungen konnten von existierenden Routing-Protokolle wie OSPF oder RIP nicht erfüllt werden. Die Implementierung von RPL im Betriebssystem Contiki wird ContikiRPL genannt \citep{tsiftes10rpl}.

Sehr wichtig für die weitere Verbreitung von 6LoWPAN ist die Interoperabilität von verschiedenen Implementierungen. \textcite{ko11beyond} haben zwei unabhängige Implementierungen, Contiki und TinyOS, zusammen getestet. Die Interoperabilität zwischen beiden war gegeben, allerdings hatten kleine Unterschiede im jeweiligen Protokollstack einen Einfluss auf die Gesamtsystemleistung. Neben der Interoperabilität ist die Leistungsfähigkeit bei dem Zusammenspiel verschiedener Implementierungen von Bedeutung.

Als weitere Schwierigkeit kommt hinzu, dass viele Implementierungen den Funkempfänger so oft wie möglich ausschalten. Im Betriebssystem Contiki steht diese Funktionalität unter dem Namen ContikiMAC zur Verfügung. So kann der Energieverbrauch um bis zu 80\% reduziert werden \citep{dunkels11contikimac}. Das gesteuerte Ausschalten des Funkempfängers hat allerdings einen markanten Einfluss auf das Kommunikationsverhalten im Netzwerk. Weil keine Spezifikationen oder Standards für diese Funktionalität existieren, verhalten sich Implementierungen sehr verschieden. \textcite{dunkels11adhoc} beschreiben, dass das Zusammenspiel von Kommunikationstechnologie und Ausschaltverhalten des Funkempfängers ein wichtiges Forschungsgebiet für das Internet-of-Things ist.

%Vergleich mit Geschichte des Elektromotors:
%http://www.udo-leuschner.de/rezensionen/rf9809dittmann.htm
%Entwicklung des elektrischen Netzes -> elektrische Energie war überall im Haushalt verfügbar. Dadurch konnte der Elektromotor quasi die alleinige Rolle des mechanischen Energielieferanten im Haushalt übernehmen. Am Anfang wurden große Elektromotoren verkauft, an die verschiedene Geräte angeschlossen werden konnten. Mittlerweile hat jedes Gerät einen kleinen Elektromotor. Das Bewusstsein seiner Existenz ist für den Verbraucher dadurch aber noch gesunken, die Technik verschwindet im Hintergrund. Schon seit Anfang der Siebziger Jahre besaß jeder Haushalt 15 bis 20 Elektromotoren.
%-> Die Technik muss in den Hintergrund treten. Darf quasi nicht bemerkt werden. Dann kann sie erfolgreich sein und sich durchsetzen.

%vermutlich wird es immer eine große Mischung aus Technologien geben.

%sonstige Vorschläge zur weiteren Forschung
