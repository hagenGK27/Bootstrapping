% !TEX root = ../Thesis.tex
%%
%%  Hochschule für Technik und Wirtschaft Berlin --  Abschlussarbeit
%%
%% Kapitel 5 - Tests
%%
%%

\chapter{Conclusion} \label{Conclusion}
A wide range of procedures and methods are employed in the field of statistics. While many of these techniques are more precise and comprehensive, it can be argued that bootstrapping is capable of achieving a level of statistical analysis that other methods are unable to match. It generates a substantial statistical analysis from a mere initial dataset. Furthermore, bootstrapping is relatively straightforward to comprehend and implement. The essential requirement is a computer with sufficient processing power to generate the requisite number of bootstraps and to perform the requisite mathematical operations. It is therefore my contention that bootstrapping represents one of the most powerful methods available for the analysis of small groups, particularly in the medical field, and the drawing of logical conclusions from them.
