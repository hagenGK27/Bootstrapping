%%
%%  Hochschule für Technik und Wirtschaft Berlin --  Abschlussarbeit
%%
%%  Hauptdokument
%%
%%%%%%%%%%%%%%%%%%%%%%%%%%%%%%%%%%%%%%%%%%%%%%%%%%%

% Wichtige Pakete und Grundeinstellungen, verwendet Dokumentenklasse "scrbook" 
\documentclass[
		chapterprefix=false, 
		a4paper, 
		twoside, 
		parskip=half, 
		listof=totoc, 
		bibliography=totoc, 
		numbers=noendperiod, 
		captions=tableheading
]{scrbook}

%% Einstellungen und Anpassungen
\input{settings/settings}  		% Import der Pakete und Stil des Dokuments
\input{settings/adjustments} 	% Weitere Pakete und Anpassungen (Sprache, Quellenverwaltung, etc.)
\input{settings/titlepage}		% Layout der Titelseite

%%%%%%%%%%%%%%%%%%%%%%%%%%%%%%%%%%%%%%%%%%%%%
% Im folgenden Bereich müssen Sie Anpassungen für das Deckblatt der Arbeit vornehmen!
%%%%%%%%%%%%%%%%%%%%%%%%%%%%%%%%%%%%%%%%%%%%%
%
%% Titel und Author 
\titel{Bootstrapping Method}
\autor{Hagen Wey}
\matrikelnr{s0575947}
%% Version und Abgabedatum
\version{0.2$\alpha$} 	%ToDo: wird derzeit noch nicht genutzt
\datum{16.07.2024}   	% Abgabedatum der Arbeit
%% Typ der Arbeit
\thesistyp{Term Paper}
%% Betreuer
\firstExaminer{Prof.~Dr. Tatiana Ermakova}
%% Fachbereich
\fachbereich{4 -- Informatik, Kommunikation und Wirtschaft --}
\studiengang{Angewandte Informatik}
%%%%%%%%%%%%%%%%%%%%%%%%%%%%%%%%%%%%%%%%%%%%%
% Ende des Bereichs für Anpassungen
%%%%%%%%%%%%%%%%%%%%%%%%%%%%%%%%%%%%%%%%%%%%%
%% Metadaten zu PDF hinzufügen
\hypersetup{
pdftitle = {\thetitel},
pdfsubject = {\thethesistyp},
pdfauthor = {\theautor},
%pdfkeywords = {Stichwort1, Stichwort2 ...} ,
pdfcreator = {LaTeX with hyperref},
pdfproducer = {pdflatex}
}
%% Pfad zu den Bildern
\graphicspath{
  {pictures/},
}
%% Current Bug-Fixes
% 2024-02 
% Error-Message ! Argument of \MT@gobble@to@nil has an extra }.
% https://tex.stackexchange.com/questions/702778/argument-of-mtgobbletonil-has-an-extra
\microtypesetup{nopatch=item}

%% Custom includes go here!!!



%%%%%%%%%%%%%%%%%%%%%%%%%%%%%%%%%%%%%%%%%%%%%
%% Start des Dokuments
%%%%%%%%%%%%%%%%%%%%%%%%%%%%%%%%%%%%%%%%%%%%%
\begin{document}

%% Deckblatt erzeugen
\maketitle

%% Inhaltsverzeichnis erstellen
\cleardoubleoddpage
\pagenumbering{Roman}
\tableofcontents \clearpage
%%%%%%%%%%%%%%%%%%%%%%%%%%%%%%%%%%%%%%%%%%%%%
%%%%%%%%%%%%%%%%%%%%%%%%%%%%%%%%%%%%%%%%%%%%%
%% In diesem Bereich müssen Sie Anpassungen für den Inhalt der Arbeit vornehmen!
%% Kurzzusammenfassung
\input{abstract_de.tex}
\input{abstract_en.tex}
\clearpage

%% Hauptteil
\cleardoubleoddpage
\pagenumbering{arabic}


\input{chapter/01_Einfuehrung.tex}
\input{chapter/02_Grundlagen.tex}
\input{chapter/03_Konzept.tex}
\input{chapter/04_Implementierung.tex}
\input{chapter/05_Tests.tex}
\input{chapter/06_Fazit.tex}

%% Anhang
\cleardoubleoddpage
\appendix
\input{chapter/99_Anhang.tex}
%%%%%%%%%%%%%%%%%%%%%%%%%%%%%%%%%%%%%%%%%%%%%
%%%%%%%%%%%%%%%%%%%%%%%%%%%%%%%%%%%%%%%%%%%%%
%%%%%%%%%%%%%%%%%%%%%%%%%%%%%%%%%%%%%%%%%%%%%

%% Abkürzungsverzeichnis
% 	Befehl: makeindex Thesis.nlo -s nomencl.ist -o Thesis.nls
%
% !TEX root = ../Thesis.tex
%%
%%  Hochschule für Technik und Wirtschaft Berlin --  Abschlussarbeit
%%
%%  Abkürzungsverzeichnis
%%
%% Vorgeplänkel nach
%% http://blog.stefan-macke.com/2006/05/03/abkurzungsverzeichnis-mit-latex/
%%%%%%%%%%%%%%%%%%%%%%%%%%%%%%%%%%%%%%%%%%


\cleardoublepage
\markboth{\nomname}{\nomname}
\printnomenclature

%% Abbildungsverzeichnis
\listoffigures \clearpage

%% Tabellenverzeichnis
\listoftables \clearpage

%% Quelltextverzeichnis
\lstlistoflistings \clearpage

%% Stichwortverzeichnis
%	Befehl: makeindex Thesis
%
\printindex \clearpage

%% Literaturverzeichnis
%	Befehl: biber Thesis
%
\printbibliography[heading=bibintoc, title={\babel{Literaturverzeichnis}{Bibliography}}]
\clearpage
%Literaturverzeichnisse getrennt nach Stichwort
%\printbibliography[heading=bibintoc, keyword={book}, title={Literaturverzeichnis}]\clearpage
%\printbibliography[heading=bibintoc, keyword={online}, title={Onlinequellen}]\clearpage
%\printbibliography[heading=bibintoc, keyword={image}, title={Bildquellen}]\clearpage

%% Erklärung zur Eigenständigkeit
\input{Eigenstaendigkeitserklaerung}

\end{document}