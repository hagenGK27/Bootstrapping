%%% Einstellungen zur Sprache und Silbentrennung im Dokument
%
% This is a document holding multiple languages.
% Switch between ENGLISH and GERMAN by commenting one of the following lines:
%\usepackage[ngerman,english]{babel} % makes ENGLISH content
\usepackage[english,ngerman]{babel} % makes GERMAN content
% this is the macro to define phrases in two languages:
\newcommand{\babel}[2]{\ifnum\pdfstrcmp{\languagename}{english}=0 {#2}\else{#1}\fi}
\newcommand{\DE}[1]{\ifnum\pdfstrcmp{\languagename}{ngerman}=0 {#1}\fi}
\newcommand{\EN}[1]{\ifnum\pdfstrcmp{\languagename}{english}=0 {#1}\fi}
% example:   \babel{Deutscher Text}{english text}
% example:   \DE{deutscher Text}
% example:   \EN{english text}
%%%%%%%%%%%%%%%


%%% Erstellung von Literaturverzeichnissen mit Biblatex (Biber/Bibtex)
% wählen Sie hier die für Ihr Latex-System nötige Literaturverarbeitung:
\usepackage[style=alphabetic, backend=biber,   natbib=true]{biblatex}  % Verarbeitung mit biber
%\usepackage[style=alphabetic, backend=bibtex, natbib=true]{biblatex}  % Verarbeitung mit bibtex
%
% Bibtex-Datei mit den Quellenangaben zur Arbeit
\addbibresource{references/references.bib}


%%% Abkürzungsverzeichnis
\usepackage[intoc]{nomencl}
\let\acro\nomenclature
\renewcommand{\nomname}{\babel{Abkürzungsverzeichnis}{Abbreviations}}
\setlength{\nomlabelwidth}{.25\hsize}
\renewcommand{\nomlabel}[1]{#1 \dotfill}
\setlength{\nomitemsep}{-\parsep}
\makenomenclature

%%% Überschriften auch in Times-Roman setzen
\addtokomafont{disposition}{\rmfamily}

%%% Kapitelnummer und Titel auf einer Zeile (erfordert die chapterprefix=false Option in class)
\renewcommand*{\chapterformat}{%
  \mbox{\chapapp~\thechapter:\enskip}%
}

%%% Farbdefinitionen
%  https://corporatedesign.htw-berlin.de/schrift-farbe/markenfarben/
\definecolor{HKS66}{RGB}{118,185,0}          %% HKS51, HTW-Grün
\definecolor{HKS47}{RGB}{0,130,209}      %% HKS47, HTW-Blau
\newcommand{\headcolor}{HKS66}
\newcommand{\alertcolor}{HKS47}
% Zusätzliche Farben
\definecolor{darkgreen}{RGB}{0,100,0}


%%% Stichwortverzeichnis 
\usepackage{imakeidx}
\makeatletter
\makeindex[columns=2, title=\babel{Stichwortverzeichnis}{Index}, options= -s settings/indexstyle.ist, intoc]
\makeatother
\indexsetup{level=\chapter*,toclevel=chapter}
% Pluszeichen in der Referenz beim Zitieren entfernen: [Kra+13] wird zu [Kra13]
\renewcommand*{\labelalphaothers}{}


%%% Darstellung von Quellcode incl. Syntax-Highlighting
\usepackage{listings}
\renewcommand{\lstlistlistingname}{\babel{Quelltextverzeichnis}{Listings}}
%Anpassungen zur Quellcodedarstellung
% muss bei Bedarf überschrieben werden (z.B. wenn unterschiedliche Sprachen zum Einsatz kommen)
\renewcommand{\lstlistingname}{Codeauszug}
\lstset{
	language=Java,
	numbers=left,
	columns=fullflexible,
	aboveskip=5pt,
	belowskip=10pt,
	basicstyle=\small\ttfamily,
	backgroundcolor=\color{black!5},
	commentstyle=\color{darkgreen},
	keywordstyle=\color{blue},
	stringstyle=\color{gray},
	showspaces=false,
	showstringspaces=false,
	showtabs=false,
	xleftmargin=16pt,
	xrightmargin=0pt,
	framesep=5pt,
	framerule=3pt,
	frame=leftline,
	rulecolor=\color{green},
	tabsize=2,
	breaklines=true,
	breakatwhitespace=true,
	prebreak={\mbox{$\hookleftarrow$}}
}

